\documentclass[paper=a4, fontsize=11pt, parskip=full]{scrartcl} % A4 paper and 11pt font size

\usepackage[T1]{fontenc} % Use 8-bit encoding that has 256 glyphs
\usepackage{fourier} % Use the Adobe Utopia font for the document - comment this line to return to the LaTeX default
\usepackage[english]{babel} % English language/hyphenation
\usepackage{amsmath,amsfonts,amsthm} % Math packages

\usepackage{graphicx}

\usepackage{float}

%Preamble
\usepackage{listings}
\usepackage{color}
\definecolor{javared}{rgb}{0.6,0,0} % for strings
\definecolor{javagreen}{rgb}{0.25,0.5,0.35} % comments
\definecolor{javapurple}{rgb}{0.5,0,0.35} % keywords
\definecolor{javadocblue}{rgb}{0.25,0.35,0.75} % javadoc

\lstset{language=Java,
basicstyle=\ttfamily,
keywordstyle=\color{javapurple}\bfseries,
stringstyle=\color{javared},
commentstyle=\color{javagreen},
morecomment=[s][\color{javadocblue}]{/**}{*/},
numbers=left,
numberstyle=\tiny\color{black},
stepnumber=2,
numbersep=10pt,
tabsize=4,
showspaces=false,
showstringspaces=false}


\usepackage{hyperref}
\hypersetup{
    colorlinks=true,
    linkcolor=blue,
    filecolor=magenta,
    urlcolor=cyan,
}

\usepackage{lipsum} % Used for inserting dummy 'Lorem ipsum' text into the template

\usepackage{sectsty} % Allows customizing section commands
\allsectionsfont{\centering \normalfont\scshape} % Make all sections centered, the default font and small caps

\usepackage{fancyhdr} % Custom headers and footers
\pagestyle{fancyplain} % Makes all pages in the document conform to the custom headers and footers
\fancyhead{} % No page header - if you want one, create it in the same way as the footers below
\fancyfoot[L]{} % Empty left footer
\fancyfoot[C]{} % Empty center footer
\fancyfoot[R]{\thepage} % Page numbering for right footer
\renewcommand{\headrulewidth}{0pt} % Remove header underlines
\renewcommand{\footrulewidth}{0pt} % Remove footer underlines
\setlength{\headheight}{13.6pt} % Customize the height of the header

\numberwithin{equation}{section} % Number equations within sections (i.e. 1.1, 1.2, 2.1, 2.2 instead of 1, 2, 3, 4)
\numberwithin{figure}{section} % Number figures within sections (i.e. 1.1, 1.2, 2.1, 2.2 instead of 1, 2, 3, 4)
\numberwithin{table}{section} % Number tables within sections (i.e. 1.1, 1.2, 2.1, 2.2 instead of 1, 2, 3, 4)

\setlength\parindent{0pt} % Removes all indentation from paragraphs - comment this line for an assignment with lots of text

%----------------------------------------------------------------------------------------
%	TITLE SECTION
%----------------------------------------------------------------------------------------

\newcommand{\horrule}[1]{\rule{\linewidth}{#1}} % Create horizontal rule command with 1 argument of height

\title{
\normalfont \normalsize
\textsc{University of Virginia, Department of Computer Science} \\ [25pt] % Your university, school and/or department name(s)
\horrule{0.5pt} \\[0.4cm] % Thin top horizontal rule
\huge Lab 08 - Hash Tables \\ % The assignment title
\horrule{2pt} \\[0.5cm] % Thick bottom horizontal rule
}

\author{Dr. Mark R. Floryan} % Your name

\date{\normalsize\today} % Today's date or a custom date

\begin{document}

\maketitle % Print the title

%----------------------------------------------------------------------------------------
%	Pre-Lab
%----------------------------------------------------------------------------------------

\section{Pre-Lab}

This week, you will be building a custom hash table class, and using it to perform a word search. The overview of this prelab is as follows:

\begin{enumerate}
	\item Download the provided \href{https://markfloryan.github.io/dsa1/labs/lab08%20-%20Hashing/code/hashing.zip}{starter code}.
	\item Implement the HashTable.java class.
	\item Test the correctness of your hash table using the provided tester.
	\item Implement the word search solver by using your hash table implementation.
	\item Optimize your hash table to run as quickly as possible on the large word search grids provided
	\item \textbf{FILES TO DOWNLOAD:} \href{https://markfloryan.github.io/dsa1/labs/lab08%20-%20Hashing/code/hashing.zip}{hashing.zip}
	\item \textbf{FILES TO SUBMIT:} lab08.zip
\end{enumerate}

%------------------------------------------------

\subsection{HashTable.java}

First, you will implement a hash table class. This class has two generic arguments K (the key) and V (the value). To handle these, we will have our table internally store HashNode objects. This code is provided to you and a summary is shown below:

\begin{lstlisting}
public class HashNode<K,V> {
	private K key; private V value;
	
	//Constructor
	public HashNode(K key, V value);
	
	//HashNodes are "equal" if both keys are equal
	public boolean equals(Object other) {
		return this.key.equals(((HashNode<K,V>)other).key);
	}
	
	public K getKey() {return this.key;}
	public V getValue() {return this.value;}
	public void setValue(V newValue) { this.value = newValue; }
}
\end{lstlisting}

Notice that two HashNode objects are considered to be "equal" to each other if the key in each is equal. Thus, you can use the provided HashNode.equals() method to examine equality between HashNode objects directly.

Using this object, you will implement the HashTable.java class which implements the Map interface shown below. You may use \emph{any collision resolution strategy} you would like when implementing this class. 

\begin{lstlisting}
public interface Map<K,V> {

	public void insert(K key, V value);
	public V retrieve(K key);
	public boolean contains(K key);
	public void remove(K key);
}
\end{lstlisting}

Once you have implemented your HashTable class, there is a provided tester that will test some of the operations of the table to ensure they work. \textbf{This tester IS NOT a thorough test, but rather a sanity check that the table seems to be working}. You should write your own tests as well to ensure your table is working correctly.

\subsection{Word Search Solver}

Next, you will be writing some code within the WordPuzzleSolver.java class. This class reads in a two-dimensional array of characters, and looks for valid English words within the puzzle. (see \href{https://en.wikipedia.org/wiki/Word_search}{Wikipedia} for more details). You will be provided with two input files, a \textbf{dictionary text file} and a \textbf{grid text file}. The dictionary file enumerates a list of valid English words. Your goal is thus to find each of these words in the word puzzle. The grid text file specifies the number of rows and cols in the grid and then enumerates the characters of the grid all on a single line.

Some other notes / requirements about this part of the lab:

\begin{enumerate}
	\item You should instantiate your hash table and insert each of the words from the dictionary. In order to this, you will need to read input from a file. See the course \href{https://markfloryan.github.io/dsa1/java/javaCheatSheet/javaCheatSheet.pdf}{java cheat sheet} for an example of how to read from a file.
	\item You need to check words that start at ANY starting location in the grid, moving in any of the eight directions (North, NorthWest, etc.), and check any words of length $3 \leq n \leq 22$. Make sure you do not move off the end of the grid when looking up words.
	\item Your output should match the format shown in the provided starter project under \emph{input/3x3.out.txt}. The order the words are printed is not important, but would be nice to make grading easier.
	\item the \emph{input/} folder provides several grids and dictionaries to test your code with. Your code should be correct for all combinations and also be reasonably fast (less than 10 seconds on the largest grids). How fast can you make it?
\end{enumerate}
%------------------------------------------------


%----------------------------------------------------------------------------------------
%	In-Lab
%----------------------------------------------------------------------------------------
\newpage
\section{In-Lab}

The goal of this in-lab is to continue practicing with hash tables in a laid back environment. As usual, you will:

\begin{enumerate}
	\item Get into small groups of two
	\item The TAs will present you with some programming challenges regarding trees. These will not be graded, but you are required to attend and to participate.
	\item You may think about the challenges before lab below if you'd like, but we highly recommend that you not solve them ahead of time.
	\item The TAs will give you time to solve each problem and lead you in sharing solutions with one another.
\end{enumerate}

\subsection{Hash Tables}

The TAs will lead you in going through the following challenges. It is ok if you do not get through each of these.

\begin{enumerate}
	\item Write a method that takes two String parameters $s1$ and $s2$. The method returns true iff $s2$ and $s1$ are anagrams of one another. Your method should use a hash table, must be $\Theta(n)$, and can only loop over each String at most once.
	\item Write a method that takes in an array of ints as a parameter. Your method should return true iff there exists four distinct elements in the array such that $a$, $b$, $c$, and $d$ such that $a \neq b \neq c \neq d$ and $a+b = c+d$. Your solution must be $O(n^2)$.
	\item Consider a list of $n$ integers (passed as a parameter) and an integer $k$ (also passed by parameter). Write a method that returns a list of all contiguous windows in the array of exactly $k$ adjacent elements AND also prints out the number of unique integers in each window. Your solution should be $O(nk)$.
\end{enumerate}

If you have coded up all of these and they work, then show the TA and you may leave early.


%------------------------------------------------


%----------------------------------------------------------------------------------------
%	Post-Lab
%----------------------------------------------------------------------------------------
\newpage
\section{Post-Lab}

The goal of this post-lab is to produce a report analyzing optimizations to your hash table implementation. 

You will perform some experiment(s) by doing the following:

\begin{enumerate}
	\item Experiment 1: Implement more than one of the collision resolution strategies shown in class. For each, count the number of collisions that occur and also record the time on various input to your word puzzle solver. Report which strategies had the most/fewest collisions, and how slow/fast the strategies were.
	\item Experiment 2: Brainstorm and implement some kind of optimization to your hash table / word puzzle solver. Describe your optimization and how it works. Report how much faster it makes your word puzzle solver run compared to the best approach from experiment 1 above. How fast were able to make your code run?
	\item \textbf{FILES TO DOWNLOAD:} None
	\item \textbf{FILES TO SUBMIT:} PostLabEight.pdf
\end{enumerate}

\subsection{Report}

Summarize your experiment and your findings in a report. Make sure to adhere to these general guidelines:

\begin{itemize}
	\item Your submission MUST BE a pdf document. You will receive a zero if it is not.
	\item Your document MUST be presented as if submitted to a professional publication outlet. You can use the \href{https://markfloryan.github.io/dsa1/labs/WordPaperTemplate.zip}{template} posted in the course repository or follow \href{https://www.springer.com/us/computer-science/lncs/conference-proceedings-guidelines}{Springer's guidelines for conference proceedings}.
	\item You should write your report as if it is original novel research.
	\item The grammar / spelling / professionalism of this document should be sound.
	\item When possible, do not use the first person. Instead of "I ran the code 60 times", use "The code was executed 60 times...".
\end{itemize}

In addition to the general guidelines above, please follow the following rough outline for your paper:

\begin{itemize}
	\item \textbf{Abstract}: Summarize the entire document in a single paragraph
	\item \textbf{Introduction}: Present the problem, and provide details regarding the algorithms you implemented (especially the "hybrid" algorithm).
	\item \textbf{Methods}: Describe your methodology for collecting data. How many executions, which inputs, when did the timer start/stop, etc.
	\item \textbf{Results}: Describe your results from your execution runs.
	\item \textbf{Conclusion}: Interpret your results. Which algorithm/approach was fastest in each situation? Did the fastest algorithm change? If so, why? Does the performance you see match the theoretical runtimes of the algorithms? Why or why not?
\end{itemize}

Lastly, your paper MUST contain the following things:

\begin{itemize}
	\item A table (methods section) summarizing the experiments and how many execution runs were done in each group.
	\item At least one table (results section) summarizing the results of all of your experiments.
	\item Some kind of graph visualizing the results of the table from the previous bullet.
\end{itemize}

%------------------------------------------------

%----------------------------------------------------------------------------------------

\end{document}


%----------------------------------------------------------------------------------------
%----------------------------------------------------------------------------------------
%----------------------------------------------------------------------------------------
%----------------------------------------------------------------------------------------
%----------------------------------------------------------------------------------------
%----------------------------------------------------------------------------------------


%WORKS CITED:

%%%%%%%%%%%%%%%%%%%%%%%%%%%%%%%%%%%%%%%%%
% Short Sectioned Assignment
% LaTeX Template
% Version 1.0 (5/5/12)
%
% This template has been downloaded from:
% http://www.LaTeXTemplates.com
%
% Original author:
% Frits Wenneker (http://www.howtotex.com)
%
% License:
% CC BY-NC-SA 3.0 (http://creativecommons.org/licenses/by-nc-sa/3.0/)
%
%%%%%%%%%%%%%%%%%%%%%%%%%%%%%%%%%%%%%%%%%

%----------------------------------------------------------------------------------------
%	PACKAGES AND OTHER DOCUMENT CONFIGURATIONS
%----------------------------------------------------------------------------------------
